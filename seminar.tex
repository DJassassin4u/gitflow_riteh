\documentclass[10pt]{beamer}

\usetheme[progressbar=frametitle]{metropolis}
\usepackage{appendixnumberbeamer}
\usepackage{graphicx}	
\usepackage{booktabs}
\usepackage[scale=1]{ccicons}

\usepackage{pgfplots}
\usepgfplotslibrary{dateplot}

\usepackage{xspace}
\newcommand{\themename}{\textbf{\textsc{metropolis}}\xspace}

\title{Seminar}
\subtitle{Gitflow}
% \date{\today}
\date{}
\author{Dorian Janžetić, Antonio Babić, Antonio Puhanić}
\institute{Tehnički Fakultet Rijeka}
% \titlegraphic{\hfill\includegraphics[height=1.5cm]{logo.pdf}}

\begin{document}

\maketitle

\section{Uvod}
\begin{frame}{Gitflow}

Gitflow je posebna vrsta workflow-a.

Dizajniran je tako da striktno definira početne grane na kojemu se cijeli projekt bazira.

Pruža framework koji je efikasan za velike projekte, ili projekte koji imaju vremenski rok.

Stirktno definira ulogu svake grane.
\end{frame}
\begin{frame}{Gitflow}
Gitflow korsti sve normalne funkcije git-a uz svoje nadogradnje.

U gitflow projektima se definira pojedinaćna grana za sve.

Npr. \textit{bugfix},\textit{hotfix}, \textit{feature} i sl.

Glavna grana na kojoj se radi je \textit{develop} i na nju se nadovezuju ostale grane poput vec navedenih.

Kada je projekt spreman, \textit{develop} grana se spaja u \textit{master} granu

\end{frame}
\begin{frame}{Gitflow}
Da bi se uopće moglo raditi na projektu uz Gitflow potrebno ga je prvo instalirati.

Jer je to zapravo nadogradnja na postojeći Git.

Instalacija je jednostavna, za OSX sisteme se može koristiti naredba \textit{brew install git-low}, a za Windows OS klasičnim preuzimanjem i instaliranjem.

Za inicijaizaciju Gitflow projekta umjesto \textit{git init}, koristi se \textit{git flow init}.

Ne mijenja ništa u projektu osim što stvara unaprijed definirane grane.
\end{frame}
\begin{frame}{Gitflow grane}
Umjesto uobičajene jedne glavne \textit{master} grane, Gitflow koristi dvije.
Druga je \textit{develop} grana.

{
\setlength{\fboxsep}{1pt}
\setlength{\fboxrule}{1pt}
\fbox{\includegraphics[width=11cm, height=4cm]{master_develop}}
}

\end{frame}
\begin{frame}{Gitflow grane}
\textit{Develop} grana je ona na kojoj se odvija većina projekta, iz koje se stvaraju i vraćaju nove grane.

Za objavljivanje tj. završavanje projekta sve se spaja u \textit{develop} granu.

Zbog ovog načina workflow-a poželjno je svaku objavu ili commit tagirati.

Ovako se bilježi cijela povijest razvoja projekta ili programa.

Na repozitoriju treba postojati develop grana koju svi developeri prate.

\end{frame}
\begin{frame}{Gitflow grane}
Svaki feature,bugfix ili hotfix koji se napravi za vrijeme projekta bi trebao biti na zasebnoj grani koja se može gurnuti na repozitorij

Kada se feature završi, razlika je u tome što se spaja sa granom \textit{develop}, a ne \textit{master}.

{
\setlength{\fboxsep}{1pt}
\setlength{\fboxrule}{1pt}
\fbox{\includegraphics[width=10cm, height=4cm]{feature}}
}
\end{frame}
\end{document}